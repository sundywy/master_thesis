\chapter{Methodology}
\section{Hardware and Protocol}
The computational data for this project is obtained by using high performance cluster in High Performance Computing Center in NTU (HPC-NTU). The connection to HPC-NTU is done through SSH (Secure Shell) protocol.

\section{Software}
Tools used for this project are \emph{Simson and gnuplot}. Simson is used in pre-processing and post-processing stages, while gnuplot is used only in pre-processing stage. Complete project workflow is illustrated in figure~\ref{fig:method}.
\begin{figure}[h]
  \setlength{\unitlength}{0.6mm}
  \centering 
  \begin{picture}(80,250)\thicklines
    \put(40,220){\framebox(40,20){fsc}}
    \put(60,220){\vector(0,-20){20}}
    \put(40,180){\framebox(40,20){bls}}
    \put(60,180){\vector(0,-25){25}}
    \put(40,135){\framebox(40,20){bla}}
    \put(60,135){\line(0,-15){15}}
    \put(30,120){\line(60,0){60}}
    \put(30,120){\vector(0,-20){20}}
    \put(90,120){\vector(0,-20){20}}
    \put(10,80){\framebox(40,20){rit}}
    \put(70,80){\framebox(40,20){pxyst}}
    \put(90,80){\vector(0,-20){20}}
    \put(70,40){\framebox(40,20){gnuplot}}
    \put(-45,170){\dashbox{1}(175,80)[lt]{\bf \scriptsize Post--processing}}
    \put(-45,130){\dashbox{1}(175,30)[lt]{\bf \scriptsize Processing}}
    \put(-45,30){\dashbox{1}(175,80)[lt]{\bf \scriptsize Pre--processing}}
    \thinlines\put(0,75){\framebox(120,185)[rt]{\bf \scriptsize Simson}}
  \end{picture}
  \caption{Project workflow}
  \label{fig:method}
\end{figure}

Simson is a CFD solver package developed by Royal Institute of Technology, Sweden. It implements spectral integration technique to solve Navier -- Stokes equation for incompressible channel and boundary flows~\cite{Simson}. Simson's code is written in Fortran 77/90, and is supported by OpenMP and MPI interface to support parallel programming. The code can be used as direct numerical simulation (DNS) and large eddy simulation (LES) solver. Simson's parts used in this project are {\bf bla} (main program), {\bf rit} (program to plot solution from complete velocity fields), and {\bf pxyst} (program to plot \emph{xy-statistics}). Complete informations of softwares used are listed in table~\ref{tab:software}.
\begin{table}[h]
  \centering
  \begin{tabular}[h]{c l l}
    \hline\hline
    No. & Software & Information \\
    \hline
    1 & fsc & Calculate velocity profile \\
    2 & bls & Calculate initial velocity field \\
    3 & bla & Main processing software \\ 
    4 & rit & Plot velocity field \\ 
    5 & pxyst & Plot statistics data  \\ 
    6 & gnuplot & Used to produce graphics\\
    \hline
  \end{tabular}
  \caption{List of softwares used in the project}
  \label{tab:software}
\end{table}

 \subsection{Processing}
The processing stage is done by using main program {\bf bla}.
 
\subsection{Pre--processing}
 Main program {\bf bla} requires velocity field as input files. This file is generated by program {\bf bls}. Velocity field generated by {\bf bls} consists of a basic laminar flow and a range of different disturbances, e.g. localized disturbance, waves and random noise. In some type of flow, laminar base flow profile is not given analytically. Therefore a velocity profile file must be generated by using program {\bf fsc}.

 {\bf bla} can be run in serial mode and parallel mode. In serial mode, {\bf bla} only requires a single processor to run the simulation. To run in parallel mode {\bf bla} is supported with MPI and OpenMP interface. Since {\bf bla} is used in batch mode, it has no interactive input. It can be configured during compilation by using file {\bf par.f} and runtime by using {\bf bla.i} file.

{\bf bla} consists of \emph{subroutines} that form the main program. In this project, subroutines that will be given special attention are {\bf rparambl}, {\bf linearbl} and {\bf cwallbc}. For a complete listing of {\bf bla's} subroutine, pelase refer to Simson's User Guide~\cite{Simson}.

\subsection{Post--processing}
{\bf rit} and {\bf pxyst} are Simson's programs used in post-processing stage. {\bf rit} is used to generate various graph of velocity field in Simson. It requires velocity field  ({\bf .u}) input files.

{\bf pxyst} can be used to plot of both time and spanwise averaged data. It generates plot for both raw statistical data and derived quantities. {\bf pxyst} is also used to generate various special plots of the mean flow, such as boundary layer thickness and skin friction. To run {\bf pxyst},  statistics ({\bf .stat}) files are required  as input.

The comparison between statistics data could not be done by using {\bf pxyst}. In order to gain more insight of flow behaviour, results obtained through {\bf pxyst} are saved in the form of data file and plotted by using gnuplot.
