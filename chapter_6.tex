\chapter{Conclusions and Recommendations}
\section{Conclusions}
Drag reduction method by the application of spanwise wall oscillation is an effective method. This has been proved by the reduction of friction coefficient and turbulent intensities on the region where the wall oscillation is applied. The drag reduction is assumed caused by the bursting process caused by the sudden transvere strain~\cite{Jung}. 

From both type of wall oscillation, oscillation with spatial variation shows better performance than the one with temporal variation. This performance differences can be related to the amount of oscillation period, in which spatial variation case has more oscillation wave compared to the one with temporal variation. It is assumed that the oscillation period decrease will increase the transverse strain experienced by the flow, which will lead to more drag reduction. 

A certain pattern that could be related to the wall oscillation is found in the friction coefficient. This pattern will form a fixed pattern with the increase of wall oscillation exposure time.

An interesting observation was found during the analysis, the turbulence intensity in the lateral direction has an unusul profile. The cause of this irregularities is not known. 

This method has been patented by Quadrio and Viotti in 2009~\cite{viotti}. However, industrial application is not very simple. Many factors should be considered carefully, e.g. power needed to produce wall oscillation and other design consideration.

\section{Recommendations}
The cause of irregularities observed in lateral direction is not found. The observation of this irregularities could become a research topic. 

