\chapter{Introduction}
\section{Background}
Since the depletion of energy resources issue, many efforts have been given to invent new energy resources. Both major inventions are related to the green energy and environment preservation principle. It is believed that these two issues are closely related. 

Energy saving has also been a major research interest in aerospace engineering. Energy saving technologies are not only a breakthrough in environment preservation effort, it also closely related to the economical part of aviation industry. By utilizing these technologies, billions of expenses can be saved from the already expensive cost needed to run an aviation company.

One of the major research interest in aerospace engineering is on the study of drag reduction on aircraft body. Turbulence has been an issue that has been faced by human kind since the invention of first aircraft by Wright brothers at 1903. It is believed that turbulence statistics on aircraft body is considered as the main culprit in increase of drag. 

In general, drag reduction method can be divided into passive and active methods. Major difference between both methods is on its dynamics characteristics. Most of engineering applications adopt passive methods due to its characteristics that do not require external power. However, active methods have gained its own popularity in academical research.

The study of turbulence statistics on aircraft body has been done by many researchers throughout the years. Research mainstreams can be divided into two main categories, computational and experimental. In the past, experimental based researches has been the dominating categories in fluid dynamics. However, with the recent computer technologies development, computational based researches has gained more attention than experimental based researches. 

This project will present a study on turbulent statistics on moving boundary layer in a flat plate. The method used is \emph{Direct Numerical Simulation (DNS)}, which has gained popularity in recent years.

\section{Objective}
The objective of this project is to investigate the effect of wall oscillation on turbulent statistics on near wall region. This project is inspired by the journal published by Viotti and Quadrio~\cite{viotti}, in which the spanwise wall oscillation will be applied with space variation. A case comparison with time variation will also be done, therefore the results obtained from previous final year project~\cite{indra} will be used as comparison.

Since the emphasize of this project is the flow's turbulence statistics, the flow will be plotted only in two-dimensional fields, unlike previous project which include three-dimensional plot. To fulfil the objectives, turbulence statistics investigation done by previous project will be expanded with more statistical parameter.

\section{Scope}
The scope for this projects are:
\begin{enumerate}
\item Investigation of the effect of spanwise wall oscillation with space variation on turbulent statistics
\item Investigation of the comparison of turbulence statistics between spanwise wall oscillation with space variation and time variation
\end{enumerate}
